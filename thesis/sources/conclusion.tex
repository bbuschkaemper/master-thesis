\chapter{Conclusion}

In this thesis,
we addressed the challenge of training universal image classification models
across multiple datasets with discrepant taxonomies.
Our research focused on developing automated methods for creating universal taxonomies
that map relationships between classes across different domains
from cross-domain predictions,
enabling the training of unified models without requiring task-specific adaptations
(in contrast to popular methods like multi-head architectures).

\section{Summary of Contributions}

We explored a range of different methods for addressing the challenges of multi-dataset image classification,
building on existing work in the field of taxonomy learning and label alignment.

\subsection{Automated Taxonomy Generation Framework}

We developed a comprehensive framework for automatically generating universal taxonomies
from multiple image classification datasets.
Building upon the method of Bevandic et al.~\cite{bevandic_automatic_2022,bevandic_weakly_2024},
we adapted their semantic segmentation approach for image classification by:

\begin{itemize}
    \item Creating a cross-domain graph generation algorithm that uses neural network predictions
          to identify relationships between classes from different datasets automatically
          and without manual intervention or correction.
    \item Developing four relationship selection methods
          (naive thresholding, most common foreign predictions, density thresholding,
          and relationship hypothesis) to filter relevant connections from noisy cross-domain predictions.
    \item Implementing universal taxonomy building rules that resolve conflicts
          between domains and create coherent universal class structures.
\end{itemize}

\subsection{Synthetic Dataset Generation for Evaluation}

Evaluation of cross-domain relationship graphs proves to be a difficult task due to the lack of ground truth data.
To address this issue, we created a novel synthetic dataset generation framework that:

\begin{itemize}
    \item Defines atomic concepts as building blocks for synthetic classes.
    \item Uses probabilistic sampling to create realistic domain variations
          with controlled complexity,
          providing a flexible framework to test a range of different scenarios.
    \item Generates ground truth relationships based on concept overlap,
          allowing precise evaluation of taxonomy generation methods.
    \item Includes domain-shifted variants to simulate realistic cross-domain challenges.
\end{itemize}

This synthetic framework provided crucial ground truth data for evaluating relationship selection methods,
something that existing real-world datasets could not offer reliably.
The framework is built modularly to allow easy extension and adaptation for future research
and new methods.

\subsection{Comprehensive Evaluation Metrics}

We introduced specialized metrics for comparing predicted and ground truth taxonomies:

\begin{itemize}
    \item Edge Difference Ratio (EDR) to measure overall relationship accuracy
          while considering edge weights.
    \item Precision, recall, and F1 scores adapted for relationship graphs.
\end{itemize}

\subsection{Universal Model Learning Architecture}

We designed and implemented a universal learning system that:

\begin{itemize}
    \item Creates mapping matrices from universal taxonomies to convert domain-specific labels
          into universal class targets.
    \item Uses discrete probability distributions as training targets
          instead of traditional one-hot encodings.
    \item Employs cross-entropy loss for probability distributions
          to handle multi-target relationships.
    \item Enables inference across multiple domains through a single unified model.
\end{itemize}

While this learning system is implemented for the ResNet model architecture,
it can easily be adapted for other architectures as well,
making it reusable across different model types.

\section{Key Findings}

Our experimental evaluation on both synthetic and real-world datasets
revealed several important insights about the effectiveness of our approach:

\subsection{Universal Model Effectiveness}

Our universal models demonstrated promising results,
consistently outperforming domain-specific baseline models on individual datasets.
We can conclude that leveraging a multi-domain training approach
provides significant benefits compared to traditional, single-domain training.

In contrast to existing work,
our approach can train on any number of datasets simultaneously,
making it highly scalable and adaptable to various multi-domain scenarios.

\subsection{Feature Space Representation Quality}

t-SNE visualization of universal model representations revealed
meaningful clustering patterns that reflected the underlying taxonomy structure.
We observed both multi-domain and domain-specific clusters,
indicating that the model successfully learned to identify both shared concepts
across different datasets and domain-specific classes.

\subsection{Synthetic Evaluation Framework Validation}

The synthetic dataset generation framework proved essential
for reliable evaluation of taxonomy generation methods.
The framework successfully provided ground truth data
that was previously unavailable for real-world datasets,
enabling precise quantitative assessment of relationship selection approaches.

\section{Limitations and Challenges}

Despite the promising results, our work encountered several limitations
that highlight areas requiring further investigation:

\subsection{Relationship Selection Method Inconsistency}

No single relationship selection method consistently outperformed the others across all scenarios.
The performance varied significantly depending on the specific dataset combination
and domain characteristics,
indicating that the current approaches lack robustness across diverse scenarios.

\subsection{Parameter Sensitivity}

The performance of relationship selection methods showed significant sensitivity to parameter choices,
and no single parameter configuration worked optimally across all dataset combinations.
This parameter dependency limits the practical applicability of the methods
without extensive hyperparameter tuning for each new dataset combination.

\subsection{Evaluation Ground Truth Limitations}

Creating reliable ground truth taxonomies for real-world datasets proved challenging.
WordNet-based approaches suffered from semantic relationships
that didn't align with visual similarities,
while simple datasets like SVHN-MNIST lacked the complexity needed for meaningful evaluation.

Our synthetic dataset generation framework, while a significant improvement,
still faced challenges in perfectly capturing the complexities of real-world scenarios.
The domain-shifted synthetic datasets better reflected real-world challenges
than original variants but remained imperfect proxies for actual cross-domain relationships.

\section{Future Work}

Based on our findings and limitations, several promising research directions emerge:

\subsection{Adaptive Relationship Selection}

Future work should investigate hybrid relationship selection methods
that can adapt to individual dataset characteristics
and automatically adjust parameters based on domain properties.
This could address the current inconsistency issues
and improve robustness across diverse scenarios.

\subsection{Enhanced Ground Truth Generation}

Research into creating more realistic ground truth datasets
that better capture the complexities of real-world cross-domain relationships is needed.
This could involve improved synthetic data generation techniques
or semi-supervised approaches that leverage human expertise more effectively.

\subsection{Comprehensive Universal Taxonomies}

Training universal models on single, comprehensive taxonomies
that span multiple datasets (both general-purpose and specialized)
represents an interesting avenue for future research.
Such an approach could potentially unlock the full benefits
of large-scale multi-domain learning by training on
a huge amount of data from diverse sources simultaneously
and automatically discovering relationships between all classes.

\section{Final Remarks}

This thesis presented a comprehensive approach to the challenge of multi-dataset image classification
through automated universal taxonomy generation.
Our work demonstrates that it is possible to automatically discover
meaningful relationships between classes across different domains
and leverage these relationships to train effective universal models.

The combination of our automated taxonomy generation framework,
synthetic evaluation methodology,
and universal learning architecture provides a solid foundation
for future research in cross-domain image classification.
While challenges remain, particularly in relationship selection consistency
and ground truth evaluation,
our results show clear benefits of multi-domain training approaches
over traditional single-domain methods.

The scalability and adaptability of our approach,
combined with its ability to work with any number of datasets simultaneously,
positions it as a valuable contribution to the field of multi-domain machine learning.
As the number of available image classification datasets continues to grow,
methods like ours will become increasingly important
for leveraging the full potential of diverse visual data sources.

% TODO Eigenständigkeitserklärung