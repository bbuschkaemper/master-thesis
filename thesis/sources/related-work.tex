\chapter{Related Work}


We will base our method of finding shared concepts between datasets on works
in the field of label-alignment and taxonomy learning.
Creating connections between datasets is an important and fundamental task
in the field of machine learning: Many datasets are created for specific tasks
and may not be directly comparable to each other. Also, inconsistency in labeling
and data collection practices can further complicate the integration of datasets.
Methods to create connections between datasets can help to align labels, improve
data quality, and facilitate knowledge transfer between models trained on different
datasets.

The most obvious and straightforward approach to align labels across datasets
is a fully manual mapping of labels from one dataset to another.
This method is time-consuming and error-prone, especially for large datasets.
Also, it can lead to further inconsistencies if multiple annotators are involved
or if the datasets are updated independently over time~\cite{bordea_semeval-2016_2016,jurgens_semeval-2016_2016,yang_literature-driven_2013}.

An improvement over this manual approach is a hybrid method that works
automated, but involves manual corrections.
For example, Firmani et al.~\cite{firmani_building_2024} proposes a weakly supervised
approach where domain experts answer targeted questions about the datasets
to guide the automated label alignment process.
This method tries to balance automation and human expertise, leveraging the strengths of both.

With the rise of large language models (LLMs), new methods have emerged
that replace the human domain experts with automated systems.
These systems leverage the capabilities of LLMs to understand and align labels
across datasets without the need for human input,
creating a pseudo-hybrid approach where the human role is replaced with LLMs~\cite{kargupta_taxoadapt_2025,chen_prompting_2023,gunn_creating_2024}.


For fully automated, non-LLM-based approaches, there are several different options available:
\begin{itemize}
    \item \textbf{WordNet} is a lexical database that groups English words into sets of synonyms called synsets, providing short definitions and usage examples.
          It can be used to find relationships between words and concepts, making it a valuable resource for aligning labels across datasets.
          However, as we will later see in \ref{sec:wordnet_synsets}, this method is error-prone and not suitable for all datasets.
    \item \textbf{Cross-Domain Classification} methods use domain models trained on a specific dataset to create cross-domain predictions.
          For example, a dataset \textit{A} can have a class \textit{A:vehicle} while a dataset \textit{B}
          has a class \textit{B:car}. By predicting an image from class \textit{B:car} using a model trained on dataset \textit{A}
          as \textit{A:vehicle}, we can collect cross-domain co-occurrences and build a mapping between the two datasets~\cite{uijlings_missing_2022,bevandic_automatic_2022,bevandic_weakly_2024}.
\end{itemize}

We will base our taxonomy construction algorithm on the works of Bevandic et al.~\cite{bevandic_automatic_2022,bevandic_weakly_2024}.
Our method will be adapted to shift the focus from aligning labels in a universal cross-domain taxonomy
towards building relationships between classes in different datasets.
Relationships will not, as in the original work, define a hierarchical structure,
but rather point out shared attributes between classes.
